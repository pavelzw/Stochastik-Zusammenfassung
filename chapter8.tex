\section*{Bedingte Wahrscheinlichkeiten}

\begin{karte}{Bedingte Wahrscheinlichkeit}
    Sei \( (\Omega, \P) \) ein diskreter W-Raum.\\
    Für \( A, B \subset \Omega \) mit \( \P(B) > 0 \) setzt man 
    \[ \P(A|B) := \frac{\P(A\cap B)}{\P(B)}. \]
    Man nennt diese Zahl bedingte Wahrscheinlichkeit von \(A\) 
    unter der Bedingung \(B\).\\
    \( \P_B := \P(\;\cdot\; | B) \) ist ein W-Maß.
\end{karte}

\begin{karte}{Multiplikationsformel}
    Es seien \( A_1, \ldots A_n \) Ereignisse mit 
    \( \P(A_1 \cap \ldots \cap A_n) > 0 \). 
    Dann gilt 
    \[ \P(\bigcap_{j=1}^n A_j) = \P(A_1) \P(A_2 | A_1) 
    \cdots \P(A_n | A_1 \cap\ldots \cap A_{n-1}). \]
\end{karte}

\begin{karte}{Formel der totalen Wahrscheinlichkeit, Bayes-Formel}
    Seien \( A_1, A_2, \ldots \) eine Zerlegung von \( \Omega \) 
    und \(B\subset \Omega\). Dann gilt 
    \begin{enumerate}
        \item \( \P(B) = \sum_{j=1}^\infty \P(B|A_j) \P(A_j) \). (Formel der totalen Wahrscheinlichkeit)
        \item Falls \( \P(B) > 0 \), so gilt 
        \[ \P(A_k | B) = \frac{ \P(A_k) \P(B | A_k) }{ \sum_{j=1}^\infty \P(B|A_j) \P(A_j) }. 
        \text{ (Bayes-Formel)} \]
    \end{enumerate}
\end{karte}