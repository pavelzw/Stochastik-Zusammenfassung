\section*{Wartezeitverteilung}

\begin{karte}{Geometrische Verteilung}
    Eine Zufallsvariable \( X \) mit Werten in \( \N_0 \)
    hat eine geometrische Verteilung (\( X \sim \Geo(p) \)), falls 
    \[ \P(X=k) = (1-p)^k p, \quad k\in\N_0. \]
    Wenn \( X \sim \Geo(p) \) gilt, dann gilt auch 
    \begin{enumerate}
        \item \( \E X = \frac{1-p}{p} \).
        \item \( V(X) = \frac{1-p}{p^2} \).
    \end{enumerate}
\end{karte}

\begin{karte}{Gedächtnislosigkeit von \( \Geo(p) \)}
    Für \( X \sim \Geo(p) \) gilt 
    \[ \P(X = m+k \;|\; X \geq k ) = \P(X = m), \quad k,m\in\N_0. \tag{\(*\)} \]
    Ist \( X \) eine \( \N_0 \)-wertige Zufallsvariable mit (\(*\)) 
    und \( 0 < \P(X = 0) < 1 \), dann gilt \( X \sim \Geo(p) \) 
    für ein \( p\in (0,1) \).
\end{karte}

\begin{karte}{Die negative Binomialverteilung}
    Die Zufallsvariable \( X \) besitzt eine \textit{negative Binomialverteilung} 
    mit Parametern \( r \) und \( p \) (\( r\in \N, 0<p<1 \)), schreibe 
    \( X \sim \Nb(r,p) \), falls gilt: 
    \[ \P(X = k) = \binom{k + r - 1}{k} p^r (1-p)^k, \quad k \in \N_0. \]
    Falls \( X \sim \Nb(r,p) \), so gelten:
    \begin{enumerate}
        \item \( \E X = r \cdot \frac{1-p}{p} \).
        \item \( V(X) = r \cdot \frac{1-p}{p^2} \)
    \end{enumerate}
\end{karte}

\begin{karte}{Negativer Binomialkoeffizient}
    Es gilt 
    \[ \binom{k+r-1}{k} = \frac{ (k + r - 1)\cdots (r + 1) r }{k!}
    = (-1)^k \binom{-r}{k}. \]
\end{karte}

\begin{karte}{Additionsgesetz negative Binomialverteilung}
    Es seien \( X \) und \(Y\) unabhängige Zufallsvariablen 
    mit \( X \sim \Nb(r,p) \) und \( Y \sim \Nb(r,p) \). 
    Dann gilt 
    \[ X + Y \sim \Nb(r + s, p). \]
\end{karte}