\section*{Hypergeom. und Binomialverteilung}
 \begin{karte}{Hypergeometrische Verteilung}
	Gegeben sei eine Urne mit $r$ roten und $s$ schwarzen Kugeln 
	$\set{1,\dotsc,r,r+1,\dotsc,r+s}$. 
	Nun wird $n$-mal gezogen ohne Zurücklegen. Setze:
 	\begin{itemize}
		\item $\Omega := \set{(a_1,\dotsc,a_n) \in [r+s]^n 
		\;|\; a_i \neq a_j \text{ für } i \neq j} \quad 
		A_j := {(a_1,\dotsc,a_n) \;|\; a_j \leq r}$
		\item $\P$ GV auf $\Omega$, $\displaystyle 
		X:= \sum_{j=1}^{n} \1_{A_j}$ Anzahl der roten Kugeln
 	\end{itemize}
	Die Verteilung von $X$ heißt dann \textit{Hypergeometrische Verteilung} 
	$X \sim \Hyp(n,r,s)$. Dann gilt
 	\begin{itemize}
 		\item $\E X = n \cdot \frac{r}{r+s}$
 		\item $\P(X=k) = \frac{\binom{r}{k} \cdot \binom{s}{n-k}}{\binom{r+s}{n}}$
 	\end{itemize}
 \end{karte}

\begin{karte}{Binomialverteilung}
	$n$-maliges Ziehen aus $r+s$ Kugeln mit Zurücklegen. 
	$\Omega := \per{r+s}{n}$, $\P$ GV, 
	$A_j:=\set{j\text{-te Kugel ist rot}}, \displaystyle 
	X := \sum_{j=1}^{n} \mathds{1}_{A_j} $. Dann gilt:
	\begin{itemize}
		\item $\E X = n \cdot \frac{r}{r+s}$
		\item $\P(X=k)=\binom{n}{k}p^k(1-p)^{n-k} \text{, wobei } p = \frac{r}{r+s}$ 
	\end{itemize}
	Dies lässt sich verallgemeinern: Seien $p \in [0,1], n \in \N$. 
	Eine ZV $X$ heißt \textit{binomialverteilt} $X \sim \Bin(n,p)$, falls 
	$$\P(X=k)=\binom{n}{k}p^k(1-p)^{n-k} \text{.}$$
\end{karte}