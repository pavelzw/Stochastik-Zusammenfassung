\section*{Mehrstufige Experimente}

\begin{karte}{Modellierung mehrstufiger Experimente}
	Sei \( \Omega = \Omega_1 \times \cdots \times \Omega_n \) diskret und 
	\( \omega = (a_1, \ldots, a_n) \in \Omega \).\\
	Die Startverteilung ist wie folgt definiert: \\
	\( \abb{p_1}{\Omega_1}{[0,1]} \), es gilt \( \sum_{\omega\in \Omega_1} p_1(\omega) = 1 \).\\
	Die Übergangswahrscheinlichkeiten sind wie folgt definiert: 
	\[ p_2(a_2 \;|\; a_1) \geq 0, p_3(a_3 \;|\; a_1, a_2) \geq 0, \ldots, 
	p_n(a_n \;|\; a_1,\ldots,a_{n-1}) \geq 0. \]
	Es muss für \(j \geq 2\) gelten
	\( \sum_{a_j \in \Omega_j} p_j(a_j \;|\; a_1,\ldots,a_j) = 1 \).\\
	Setze für \(A \subset \Omega \)
	\[ \P(A) := \sum_{a \in A} p(a), \]
	wobei \( p(a) = p_1(a_1) p_2(a_2 \;|\; a_1) 
	\cdots p_n(a_n \;|\; a_1,\ldots a_{n-1}) \). 
	\(\P\) ist ein W-Maß auf \(\Omega\).
\end{karte}

\begin{karte}{Produkt von W-Räumen}
	Für gegebene diskrete W-Räume \( (\Omega_1, \P_1),\ldots, (\Omega_n, \P_n) \)
	heißt der durch \( \Omega := \Omega_1 \times \cdots \times \Omega_n \), 
	\( \P(A_1 \times \cdots \times A_n) := \P_1(A_1) \cdots \P_n(A_n) \)
	festgelegte W-Raum \((\Omega, \P)\) Produkt der W-Räume 
	\( (\Omega_1, \P_1),\ldots, (\Omega_n, \P_n) \).
	Man schreibt auch \( \P := \bigotimes_{j=1}^n \P_j \).
\end{karte}