\section*{Zufallsvektoren, gem. Verteilung}

\begin{karte}{Zufallsvektor, gemeinsame Verteilung}
    Es seien \( \abb{X_j}{\Omega}{\R}, j = 1,\ldots,n \) Zufallszahlen. 
    Dann heißt die durch 
    \[ X(\omega) := (X_1(\omega), \ldots, X_n(\omega)), \omega \in \Omega \]
    definierte Abbildung \( \abb{X}{\Omega}{\R^n} \) ein \(n\)-dimensionaler 
    Zufallsvektor (ZV). 
    Das durch \( M \subset \R^n, \)
    \[ \P^X(M) := \P(X^{-1}(M)) = \P(X \in M) 
    = \P(\set{ \omega : X(\omega) \in M }) \]
    definierte Wahrscheinlichkeitsmaß 
    \( \abb{\P^X}{\mathcal{P}(\R^n)}{[0,1]} \) 
    heißt Verteilung von \(X\).\\
    \( \P^{X_j} \) von \(X_j\) heißt \(j\)-te Marginalverteilung von \(X\).
\end{karte}