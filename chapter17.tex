\section*{Grenzwertsätze}

\begin{karte}{Schwaches Gesetz der großen Zahlen}
	Seien $X_1,X_2,\dotsc$ unabhängige ZVen mit gleicher Verteilung und $\E X_1 < \infty$. Sei 
	$$ \overline{X_n}:=\frac{1}{n}\sum_{j=1}^{n} X_j, \quad j \in \N.$$
	Dann gilt für jedes $\varepsilon > 0$:
	$$\lim\limits_{n \to \infty} \P^{(n)}(\abs{\overline{X_n}-\E X_1} \geq \varepsilon) = 0.$$
\end{karte}

\begin{karte}{Stochastische Konvergenz}
	Seien $Y_1,\dotsc ,Y_n$ Zufallsvariablen und $a \in \R$.
	$$ Y \plim{\P}{n \to \infty} a \quad :\Leftrightarrow \quad \lim\limits_{n \to \infty} 
	\P(\abs{Y_n-a}\geq \varepsilon) = 0 \quad \forall \varepsilon > 0$$
	Man nennt dies \textit{Konvergenz in Wahrscheinlichkeit} bzw. \textit{stochastische Konvergenz}.
	Ist $Y$ eine weitere ZV, so schreibt man:
	$$ Y \oversett{$\P$}{\undersett{$n \to \infty$}{\longrightarrow}} Y \quad :\Leftrightarrow 
	\quad \lim\limits_{n \to \infty} \P(\abs{Y_n-Y}\geq \varepsilon) = 0 \quad \forall \varepsilon > 0.$$
\end{karte}

\begin{karte}{Schwaches Gesetz großer Zahlen von Jakob Bernoulli}
	Seien $A_1,A_2,\dotsc$ unabhängig mit $\forall j \in \N :\P(A_j)=p$, so ist
	$$R_n := \frac{1}{n}\sum_{j=1}^{n}\1_{A_j} \plim{\P}{n \to \infty} p.$$
\end{karte}

\begin{karte}{Motivation des zentralen Grenzwertsatzes}
	Sei $p \in (0,1)$. Ferner seien $X_1,X_2,\dotsc$ unabhängige $\Bin(1,p)$-verteilte Zufallsvariablen. 
	Sei $S_n := \sum_{j=1}^{n}X_j, j \in \N$. \\
	\textbf{Frage:} Wie stark schwankt $S_n$ um den Erwartungswert $np$? \\
	Es gilt für eine reelle Zahlenfolge $(a_n)_{n \in \N}$:
	$$ \lim\limits_{n \to \infty}\P(\abs{S_n-np} \leq a_n) = \begin{cases}
	1 & \text{falls } \lim\limits_{n \to \infty} \frac{a_n}{\sqrt{n}} = \infty \\
	0 & \text{falls } \lim\limits_{n \to \infty} \frac{a_n}{\sqrt{n}} = 0
	\end{cases}$$
\end{karte}

\begin{karte}{Stirlingsche Formel}
	Für $n \in \N$ gilt:
	$$ n! = \sqrt{2\pi n}\cdot n^n \cdot (1+R(n)) \quad \text{ mit } \quad 0 \leq R(n)
	 \oversett{$n \to \infty$}{\longrightarrow} 0.$$
	Insbesondere gilt $n! \sim \sqrt{2\pi n}\cdot n^n \cdot$, wobei $a_n \sim b_n :\Leftrightarrow 
	\lim\limits_{n \to \infty}\frac{a_n}{b_n} = 1$.
\end{karte}

\begin{karte}{Normalverteilung}
	\begin{itemize}
		\item Sei $$\abb{\varphi}{\R}{\R}, x \mapsto \frac{1}{2\pi}\mathrm{e}^{-\frac{x^2}{2}}.$$ $\varphi$
		 heißt \textit{Dichte der standardisierten Normalverteilung} oder \textit{Gaußsche Glockenkurve}. 
		\item Setze $$\abb{\phi}{\R}{\R}, x \mapsto \int_{-\infty}^{x} \varphi(t) \; \mathrm{d}t.$$ $\phi$
		 heißt \textit{Verteilungsfunktion} der standardisierten Normalverteilung.
		\item Es gilt: $ \displaystyle \int_{-\infty}^{\infty} \varphi(t) \; \mathrm{d}t = 1$,
		 $\phi(x)+\phi(-x)=1$.
	\end{itemize}
\end{karte}

\begin{karte}{Zentraler Grenzwertsatz von de Moirre-Laplace}
	Sei $S_n \sim \Bin(n,p)$, $p \in (0,1)$, $S_n^\ast := \frac{S_n-np}{\sqrt{np(1-p)}}$. Dann gelten:
	\begin{itemize}
		\item $\displaystyle \lim\limits_{n \to \infty} \P(a \leq S_n^\ast \leq b) = \phi(b)-\phi(a), \quad a,b \in \R$
		\item $\displaystyle \lim\limits_{n \to \infty} \P(S_n^\ast \leq x) = \phi(x), \quad x \in \R$.
	\end{itemize}
\end{karte}

\begin{karte}{Zentraler Grenzwertsatz von Lindeberg-Lévy}
	Seien $X_1,X_2,\dotsc$ unabhängige ZVen mit gleicher Verteilung, wobei $\E X_1 < \infty$. 
	Sei $\mu := \E X_1, 0 < \sigma^2 := V(X_1), S_n := \sum_{j=1}^{n} X_j$, 
	$$S_n^\ast := \frac{S_n-n\mu}{\sqrt(n\sigma^2)} = \frac{S_n-\E S_n}{\sqrt{V(S_n)}}.$$
	Dann gelten:
	\begin{itemize}
		\item $\displaystyle \lim\limits_{n \to \infty} \P(a \leq S_n^\ast \leq b) = \phi(b)-\phi(a), \quad a,b \in \R$
		\item $\displaystyle \lim\limits_{n \to \infty} \P(S_n^\ast \leq x) = \phi(x), \quad x \in \R$.
	\end{itemize}
\end{karte}