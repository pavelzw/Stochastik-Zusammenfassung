\section*{Erzeugende Funktionen}

\begin{karte}{Erzeugende Funktion}
	Es sei $X$ eine $\N_0$-wertige ZV. Die durch 
	$$ g_X(t):= \sum_{k=0}^{\infty} \P(X=k) t^k, \quad t \in \R$$
	definierte Funktion heißt \textit{erzeugende Funktion von} $X$ (bzw. $\P^X$). \\
	\textbf{Bemerkungen:}
	\begin{itemize}
		\item Es gilt $g_X(1)=1$, der Konvergenzradius von $g_X$ ist also mindestens $1$.
		\item Es ist $g_X(t)=\E[t^X]$.
		\item Die Funktion $g_X$ legt die Verteilung von $X$ eindeutig fest (Identitätssatz für Potenzreihen).
	\end{itemize}
\end{karte}

\begin{karte}{Multiplikationssatz}
	Seien $X,Y$ unabhängige, $\N_0$-wertige ZVen. Dann gilt:
	$$ g_{X+Y}(t) = g_X(t) \cdot g_Y(t) \quad \forall t \in \R.$$
\end{karte}

\begin{karte}{Erwartungswerte der faktoriellen Momente}
	Sei $X$ eine $\N_0$-wertige ZV. Dann ist für $r\in \N$ die ZV $X(X-1)\cdots(X-r+1)$ genau dann integrierbar, 
	wenn $g^{(r)}(1-):= \lim\limits_{r \uparrow 1} g^{(r)}(t)$ existiert. In diesem Fall gilt:
	$$ \E[X(X-1)\cdots(X-r+1)]=g^{(r)}(1-).$$
\end{karte}